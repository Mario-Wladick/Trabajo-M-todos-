\documentclass[12pt]{article}
\usepackage[utf8]{inputenc}
\usepackage[spanish]{babel}
\usepackage{amsmath}
\usepackage{booktabs}
\usepackage{geometry}
\usepackage{natbib}
\usepackage{url}
\usepackage{longtable}
\usepackage{fancyhdr}

\geometry{a4paper, margin=2.5cm}

\pagestyle{fancy}
\fancyhf{} 
\fancyhead[L]{Repositorio: https://github.com/Mario-Wladick/Trabajo-M-todos-.git } 


\title{Definiciones: Variable, Función y Restricción}
\author{Mario Wilfredo Ramirez Puma}
\date{13 de abril de 2025}

\begin{document}
	
	\maketitle
	
	\section*{Variable: Nivel de Estrés Laboral}
	
	\subsection*{Definición}
	El estrés constituye uno de los problemas de salud más generalizado actualmente. Es un fenómeno multivariable resultante de la relación entre la persona y los eventos de su medio \citep{alfonso2015estres}.
	
	\subsection*{Función}
	En investigaciones, el nivel de estrés laboral puede actuar como:
	
	\begin{itemize}
		\item \textbf{Variable dependiente:} cuando se analiza cómo factores como la carga de trabajo, el ambiente organizacional o el liderazgo afectan el nivel de estrés del trabajador.
		\item \textbf{Variable independiente:} cuando se estudia cómo el estrés laboral influye en otras variables, como la productividad, la satisfacción laboral o la salud mental.
	\end{itemize}
	
	\subsection*{Restricciones}
	Para delimitar el estudio del estrés laboral, se pueden establecer las siguientes restricciones:
	
	\begin{itemize}
		\item \textbf{Población específica:} por ejemplo, trabajadores del sector salud que laboran en turnos nocturnos.
		\item \textbf{Antigüedad laboral:} incluir solo a empleados con más de un año en la empresa.
		\item \textbf{Instrumento de medición:} utilizar herramientas validadas, como el Cuestionario de Estrés Percibido (PSS).
	\end{itemize}
	
	\subsection*{Organización de la Variable}
	
	\begin{longtable}{@{}llll@{}}
		\toprule
		\textbf{Variable} & \textbf{Funciones} & \textbf{Indicadores} & \textbf{Instrumento} \\ \midrule
		Nivel de Estrés Laboral & Carga laboral, presión & Horas extra, tensiones con jefes & Cuestionario de Estrés Percibido (PSS) \\
		& Bienestar emocional & Ansiedad, irritabilidad & Entrevista semi-estructurada \\ \bottomrule
	\end{longtable}
	
	\bibliographystyle{apa}
	\bibliography{referencias}
	
\end{document}

